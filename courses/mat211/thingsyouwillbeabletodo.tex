\documentclass[10pt]{article}
\usepackage{amsmath}
\usepackage{amsfonts}
\usepackage{caption}
\usepackage{verbatim}
\usepackage{ulem}
\usepackage[margin=1.25in]{geometry}
\captionsetup{labelformat=empty}
\begin{document}

\setlength\parindent{0pt}

\centerline{ \large Things you will be able to do}
\vspace{1pc}

\begin{enumerate}
\itemsep0em
\item{Solve a linear system directly (when there are 3 or fewer unknown variables), and give an explicit parameterization of the set of solutions}
\item{Draw accurate diagrams of vectors, lines, and planes in $\mathbb{R}^2$ and $\mathbb{R}^3$}
\item{Convert a linear system to a matrix equation}
\item{Represent a linear system as an augmented matrix}
\item{Perform row operations (scaling, swapping, and in-place subtraction) on a matrix or augmented matrix}
\item{Perform the correct sequence of row operations on an augmented matrix leading to reduced row echelon form (Gaussian elimination)}
\item{Perform the correct sequence of row operations on a matrix leading to row echelon form (partial Gaussian elimination)}
\item{Determine an explicit parameterization of the set of solutions of a linear system using the reduced row echelon form of the augmented matrix}
\item{Write the solution of a linear system as a vector expression}
\item{Perform the correct sequence of row operations on the identity-augmented matrix leading to the matrix of the inverse transformation}
\item{Represent a change of basis as a matrix $B$ (most often, from the standard basis to a given different basis)}
\item{Represent a vector $v$ in $\mathbb{R}^n$ (given as a column vector with respect to the standard basis) as a column vector with respect to a given different basis ($Bv$)}
\item{Represent a vector $v$ given as a column vector with respect to a given basis as a column vector with respect to another basis ($Bv$)}
\item{Represent a linear map or transformation as a matrix with respect to the standard basis ($M$)}
\item{Represent a linear map or transformation as a matrix with respect to a given basis ($CMB^{-1}$ or $BMB^{-1}$)}
\item{Represent a bilinear form $\mathbb{R}^n\times\mathbb{R}^n\rightarrow\mathbb{R}$ as a matrix with respect to the standard basis ($M$)}
\item{Represent a bilinear form $\mathbb{R}^n\times\mathbb{R}^n\rightarrow\mathbb{R}$ as a matrix with respect to a given basis ($B^{T}MB$)}
\item{Compute the value (output) of a linear map or transformation when applied to a given input vector}
\item{Compute the value of a bilinear form on a pair of input vectors}
\item{Compute the length of a vector in $\mathbb{R}^n$}
\item{Compute the length of the projection of a vector onto another vector}
\item{Write the matrix of a linear transformation $\mathbb{R}^n\rightarrow\mathbb{R}^n$ that orthogonally projects onto the line spanned by a given vector (i) in an easy basis, and (ii) in the standard basis}
\item{Write the matrix of a linear transformation that orthogonally projects onto a subspace given by an orthonormal spanning set (i) in an easy basis, and (ii) in the standard basis}
\item{Write the matrix of a linear transformation that orthogonally projects onto a subspace given by an arbitrary spanning set}
\item{Write the matrix of a linear transformation that rotates in a given 2-plane by a given angle (i) in an easy basis, and (ii) in the standard basis}
\item{Represent the composition (one after the other) of two linear maps or transformations as a matrix (matrix multiplication)}
\item{Compute the determinant of a 2 by 2 or 3 by 3 matrix}
\item{Instruct a computer to compute the determinant of an $n$ by $n$ matrix}
\item{(column space) Compute a basis for the image of a linear map $\mathbb{R}^n\rightarrow\mathbb{R}^m$ (the column space of the matrix; select the columns of the original matrix corresponding to pivots)}
\item{(null space) Compute a basis for the kernel of a linear map $\mathbb{R}^n\rightarrow\mathbb{R}^m$ (use the parameterization of the solution set of the homogeneous linear system $Ax=0$, where $A$ is the matrix of the map)}
\item{(row space) Compute a basis for the image of the dual map $\mathbb{R}^{m*}\rightarrow\mathbb{R}^{n*}$ (transpose then compute a basis for the column space)}
\item{('left' null space) Compute a basis for the kernel of the dual map $\mathbb{R}^{m*}\rightarrow\mathbb{R}^{n*}$ (the kernel of the transpose matrix equation $A^{T}x=0$)}
\item{Relate the dimensions of these 4 spaces and the rank of the original matrix:
\begin{align*}
&\text{rank}=\operatorname{dim}(\text{col space})=\operatorname{dim}(\text{row space})\\
&\operatorname{dim}(\text{col space})+\operatorname{dim}(\text{null space})=n\\
&\operatorname{dim}(\text{row space})+\operatorname{dim}(\text{left null space})=m\\
\end{align*}
}
\item{Convert a list of vectors to an orthonormal basis for their span (Gram-Schmidt)}
\item{Decompose a full-rank matrix as $QR$, where the columns of $Q$ are orthonormal and $R$ is upper-triangular with positive entrires on the diagonal (remembering the moves of Gram-Schmidt)}
\item{Decompose a matrix as $LU$, where $L$ is lower-triangular and $U$ is upper-triangular, when such a decomposition exists (remembering the moves of Gaussian elimination)}
\item{Use an $LU$ decomposition to solve a linear system $LUx=b$}
\item{Compute the least squares solution of an overdetermined linear system $Ax=b$ (the actual solution of $A^{T}Ax=A^{T}b$)}
\item{Perform symbolic matrix operations: product, inverse, transpose, determinant, trace, using linearity, basis invariance, associativity, homogeneity, etc.:
\begin{align*}
&A+B=B+A\\
&(AB)C=A(BC)\\
&(AB)^{-1}=B^{-1}A^{-1}\\
&(AB)^{T}=B^{T}A^{T}\\
&(A^{-1})^{T}=(A^{T})^{-1}\\
&\operatorname{tr}(A+B)=\operatorname{tr}A+\operatorname{tr}B\\
&\operatorname{tr}(BAB^{-1})=\operatorname{tr}A\\
&\operatorname{det}(BAB^{-1})=\operatorname{det}A\\
&\operatorname{tr}(\lambda A)=\lambda\operatorname{tr}A\\
&\operatorname{det}(\lambda A)=\lambda^{\operatorname{size}A}\operatorname{det}A\\
\end{align*}}
\item{Compute the characteristic polynomial of a linear transformation}
\item{Compute the eigenvalues of a linear transformation}
\item{Compute a basis of eigenvectors for the eigenspace of a given eigenvalue of a linear transformation}
\item{Find a basis with respect to which the matrix of a given linear transformation is diagonal, if one exists, and write the matrix equation relating the matrix with respect to the standard basis to the diagonal form}
\end{enumerate}
\end{document}