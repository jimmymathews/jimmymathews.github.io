\documentclass[10pt]{article}
\usepackage{amsmath}
\usepackage{amsfonts}
\usepackage{caption}
\usepackage{verbatim}
\usepackage{ulem}
\usepackage[margin=1.25in]{geometry}
\captionsetup{labelformat=empty}
\begin{document}

\setlength\parindent{0pt}

\centerline{ \large MAT 211 Introduction to Linear Algebra}
\vspace{1pc}

\textbf{Schedule}: Monday Wednesday Thursday (Friday) 6:00-8:15pm, Main Library N4006

\begin{itemize}
\item{May \sout{27} 29 30 (31).  Monday Memorial Day holiday, Friday correction day}
\item{June 3 5 6.}
\item{June 10 12 13.}
\item{June 17 19 20 (21).  Friday correction day}
\item{June 24 26 27.  All homework due Friday}
\item{July 1 3.  Final exam on Wednesday July 3rd 6:00pm}
\end{itemize}

\textbf{Instructor}: Jimmy Mathews

\vspace{1pc}

\textbf{Office}: Math building 2-105

\vspace{1pc}

\textbf{Grades}: 25\% final exam, 75\% homework

\vspace{1pc}

\textbf{Textbook}:  The official textbook for the course is Bretscher, \emph{Linear Algebra with Applications} (4th edition).  It is important to read a textbook before and after class, but the material is standard and you may choose to learn the topics from another book.

\vspace{1pc}

\textbf{Description}: The theory of linear algebra is simple, deep, and difficult at first.  It is the content of the linear algebra course MAT 310, which also serves as a solid introduction to proof writing.  In this course we will mostly avoid theory, which will free us to emphasize utility: algorithms and computation.  The technique of linearization often reduces hard problems in engineering, the sciences, and mathematics to these computations.

The main topics in this course are common to all of these areas:
\begin{itemize}
\item{linear systems, Gaussian elimination, Cramer's rule, $LU$ factorization, matrix inverses}
\item{bases, dimension, rank}
\item{the four fundamental subspaces of a matrix}
\item{dot product, orthogonality, orthogonal projections, Gram-Schmidt algorithm, $QR$ factorization, least squares solutions}
\item{trace and determinant, characteristic polynomial, eigenvalues and eigenvectors, symmetric matrices and quadratic forms, diagonalization}
\end{itemize}

Additional topics may include: Jordan normal form, singular value decomposition, polar decomposition, difference equations and linearization of differential equations, discrete Fourier transform and signal analysis, infinitesimal transformations, tensors, inertia and angular momentum, stress/strain and elasticity, projective transformations and graphics, Lorentz transformations and relativity.

\vspace{1pc}

\textbf{Homework}: In class I will explain how to do the things listed below.  As you learn them, you should work out a well-chosen example and write it in your homework notebook, complete with a concise description in words whenever possible.  I will collect your notebooks on Monday and return them on Wednesday with integer grades from 0 to 1 for each item.  1 means the write-up is clear, correct, and complete.  0 means it is not.  You may resubmit to get 1's until Friday, June 27.  At the end of the course you will have a complete study guide for the final exam in your own handwriting.

\pagebreak

\center{\textbf{43 things}}
\begin{enumerate}
\itemsep0em
\item{Solve a linear system directly (when there are 3 or fewer unknown variables), and give an explicit parameterization of the set of solutions}
\item{Draw accurate diagrams of vectors, lines, and planes in $\mathbb{R}^2$ and $\mathbb{R}^3$}
\item{Convert a linear system to a matrix equation}
\item{Represent a linear system as an augmented matrix}
\item{Perform row operations (scaling, swapping, and in-place subtraction) on a matrix or augmented matrix}
\item{Perform the correct sequence of row operations on an augmented matrix leading to reduced row echelon form (Gaussian elimination)}
\item{Perform the correct sequence of row operations on a matrix leading to row echelon form (partial Gaussian elimination)}
\item{Determine an explicit parameterization of the set of solutions of a linear system using the reduced row echelon form of the augmented matrix}
\item{Write the solution of a linear system as a vector expression}
\item{Perform the correct sequence of row operations on the identity-augmented matrix leading to the matrix of the inverse transformation}
\item{Compute the value (output) of a linear map or transformation when applied to a given input vector}
\item{Represent a change of basis as a matrix $B$ or $P$ ($P=B^{-1}$) (most often, from the standard basis to a given different basis)}
\item{Represent a vector $v$ in $\mathbb{R}^n$ (given as a column vector with respect to the standard basis) as a column vector with respect to a given different basis ($Bv$ or $P^{-1}v$)}
\item{Represent a vector $v$ given as a column vector with respect to a given basis as a column vector with respect to another basis ($Bv$ or $P^{-1}v$)}
\item{Represent a linear map or transformation as a matrix with respect to the standard basis ($M$)}
\item{Represent a linear map or transformation as a matrix with respect to a given basis ($CMB^{-1}$ or $BMB^{-1}$ (or $P^{-1}MP$))}
\item{Represent a bilinear form $\mathbb{R}^n\times\mathbb{R}^n\rightarrow\mathbb{R}$ as a matrix with respect to the standard basis ($M$)}
\item{Represent a bilinear form $\mathbb{R}^n\times\mathbb{R}^n\rightarrow\mathbb{R}$ as a matrix with respect to a given basis ($(B^{-1})^{T}MB^{-1}$ or $P^{T}MP$)}
\item{Compute the value of a bilinear form on a pair of input vectors}
\item{Compute the length of a vector in $\mathbb{R}^n$}
\item{Compute the length of the projection of a vector onto another vector}
\item{Write the matrix of a linear transformation $\mathbb{R}^n\rightarrow\mathbb{R}^n$ that orthogonally projects onto the line spanned by a given vector (i) in an easy basis, and (ii) in the standard basis}
\item{Write the matrix of a linear transformation that orthogonally projects onto a subspace given by an orthonormal spanning set (i) in an easy basis, and (ii) in the standard basis}
\item{Write the matrix of a linear transformation that orthogonally projects onto a subspace given by an arbitrary spanning set}
\item{Write the matrix of a linear transformation that rotates in a given 2-plane by a given angle (i) in an easy basis, and (ii) in the standard basis}
\item{Represent the composition (one after the other) of two linear maps or transformations as a matrix (matrix multiplication)}
\item{Compute the determinant of a 2 by 2 or 3 by 3 matrix}
\item{Compute the solution of a determined linear system directly using Cramer's rule}
\item{(column space) Compute a basis for the image of a linear map $\mathbb{R}^n\rightarrow\mathbb{R}^m$

(select the columns of the original matrix corresponding to pivots in echelon form)}
\item{(null space) Compute a basis for the kernel of a linear map $\mathbb{R}^n\rightarrow\mathbb{R}^m$

(use the parameterization of the solution set of the homogeneous linear system $Ax=0$, where $A$ is the matrix of the map)}
\item{(row space) Compute a basis for the image of the dual map $\mathbb{R}^{m*}\rightarrow\mathbb{R}^{n*}$

(transpose then compute a basis for the column space)}
\item{('left' null space) Compute a basis for the kernel of the dual map $\mathbb{R}^{m*}\rightarrow\mathbb{R}^{n*}$

(transpose then compute a basis for the null space)}
\item{Relate the dimensions of these 4 spaces and the rank of the original matrix:
\begin{align*}
&\text{rank}=\operatorname{dim}(\text{col space})=\operatorname{dim}(\text{row space})\\
&\operatorname{dim}(\text{col space})+\operatorname{dim}(\text{null space})=n\\
&\operatorname{dim}(\text{row space})+\operatorname{dim}(\text{left null space})=m\\
\end{align*}
}
\item{Convert a list of vectors to an orthonormal basis for their span (Gram-Schmidt)}
\item{Decompose a full-rank matrix as $QR$, where the columns of $Q$ are orthonormal and $R$ is upper-triangular with positive entrires on the diagonal (remembering the moves of Gram-Schmidt)}
\item{Decompose a matrix as $LU$, where $L$ is lower-triangular and $U$ is upper-triangular, when such a decomposition exists (remembering the moves of Gaussian elimination)}
\item{Use an $LU$ decomposition to solve a linear system $LUx=b$}
\item{Compute the least squares solution of an overdetermined linear system $Ax=b$ (the actual solution of $A^{T}Ax=A^{T}b$)}
\item{Perform symbolic matrix operations: product, inverse, transpose, determinant, trace, using linearity, basis invariance, associativity, homogeneity, etc.:
\begin{align*}
&A+B=B+A &\operatorname{tr}(A+B)=\operatorname{tr}A+\operatorname{tr}B\\
&(AB)C=A(BC) &\operatorname{tr}(BAB^{-1})=\operatorname{tr}A\\
&(AB)^{-1}=B^{-1}A^{-1} &\operatorname{det}(BAB^{-1})=\operatorname{det}A\\
&(AB)^{T}=B^{T}A^{T} &\operatorname{tr}(\lambda A)=\lambda\operatorname{tr}A\\
&(A^{-1})^{T}=(A^{T})^{-1} &\operatorname{det}(\lambda A)=\lambda^{\operatorname{size}A}\operatorname{det}A\\
\end{align*}}
\item{Compute the characteristic polynomial of a linear transformation}
\item{Compute the eigenvalues of a linear transformation}
\item{Compute a basis of eigenvectors for the eigenspace of a given eigenvalue of a linear transformation}
\item{Find a basis with respect to which the matrix of a given linear transformation is diagonal, if one exists, and write the matrix equation relating the matrix with respect to the standard basis to the diagonal form}
\end{enumerate}
\end{document}