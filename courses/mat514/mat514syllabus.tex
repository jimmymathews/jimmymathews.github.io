\documentclass[12pt]{article}
\usepackage{amsmath}
\usepackage{amsfonts}
\usepackage{url}
\usepackage{caption}
\usepackage{verbatim}
\usepackage{enumerate}
\usepackage[all]{xy}
\usepackage[margin=1.25in]{geometry}
\captionsetup{labelformat=empty}
\date{July 8 2014}
\title{MAT 514 Complex Analysis}
\author{James Mathews}
\begin{document}
\maketitle
\setlength\parindent{0pt}

This course is an introduction to complex numbers and functions of a complex variable.  The theory compiles dozens of mathematical innovations in different areas into a surprisingly neat framework.

I expect the coursework to take approximately 13 hours per week outside of class time.

\begin{center}\textbf{Grades}\end{center}

Each student will receive a grade for the following written assignments, and the student's numerical final grade will be computed as a weighted average with the indicated weights:

\begin{center}
  \begin{tabular}{ l | c || r }
    assignment & dates & weight \\ \hline
    reading & new assignment every class & 0.20 \\
    homework & due first class of the week & 0.30 \\
    class exercises & varies & 0.10\\
    final exam &  August 14 2014 & 0.40
  \end{tabular}
\end{center}

\begin{center}\textbf{Reading}\end{center}

The textbook for this course is a series of online lectures which can be found at

\begin{center}\url{http://people.math.gatech.edu/~cain/winter99/complex.html}\end{center}

Please copy the following table into your notebook:

\begin{center}
  \begin{tabular}{ c | c | c | c | c | c | c }
     & & chapter & 1st reading (1) & 2nd (1) & 3rd (2) & final (0.5)\\ \hline
    Tuesday & 07/08 & & & & &\\ \hline
    Thursday & 07/10 & 1  & & & &\\ \hline
    Tuesday & 07/15 & 2  & & & &\\ \hline
    Thursday & 07/17 & 3  & & & &\\ \hline
    Tuesday & 07/22 & 4  & & & &\\ \hline
    Thursday & 07/24 & 5  & & & &\\ \hline
    Tuesday & 07/29 & 6  & & & &\\ \hline
    Thursday & 07/31 & 7  & & & &\\ \hline
    Tuesday & 08/05 & 8  & & & &\\ \hline
    Thursday & 08/07 & 9  & & & &\\ \hline
    Tuesday & 08/12 &   & & & &\\ \hline
    Thursday & 08/14 &   & & & &\\
  \end{tabular}
\end{center}

Please follow the reading suggestions:
\begin{enumerate}
\item{Plan a \emph{lot} of time for reading and re-reading.  I suggest the number of hours in parentheses in each column above, though it is not unusual to need much more time.  It is the primary way to learn mathematics!}
\item{I suggest 1 hour for each of your first readings.  For your first 2 readings: 

Get comfortable.  Print out the chapter if paper is easier.  Find space out of conversational distance from others.  Read each sentence slowly.

When an unfamiliar or problematic sentence, phrase, or symbol appears, read it again, at the same speed, then pause for 5-30 seconds.  If it is still problematic, skip it and pretend for the rest of your reading session that you believe it and that it makes sense.

Make a mental note of the number of problems you encountered.

Mark the number of hours you read in your chart (I do not require that you spend the suggested number of hours, but I do require that you report accurately).}
\item{I suggest roughly 2 hours for your 3rd reading.  For your 3rd reading, get a pencil or pen ready and prepare a section in your notebook:

Each time a definition, theorem, concept, process, or new point of view is introduced: \emph{If you understand it}, choose a word or noun-phrase that describes its content as you understand it, and a longer phrase explaining what it is.  \emph{If you don't understand it}, choose a word or phrase describing the topic, and make a special mark (for example, a *) that indicates incomplete understanding.

As a general rule, omit worked examples from your reading notes.

Mark in the margins of your text sentences, paragraphs, or sections which contain the `filler' used for explanation and example (these are essential for the purpose of reading to learn, but unneeded for the purpose of reading to remember).

Mark the number of hours you read in your chart.
}
\item{I suggest half an hour for your final reading.  For your final reading:

Read the text relatively quickly alongside your notes.

Confirm that your understanding as recorded in your notes agrees with your current understanding (otherwise update your notes).

For each outstanding topic, spend 5-10 minutes formulating a precise question to resolve it.  Record your questions.  **Questions asked in class are automatically eligible for the \emph{excellent question bonus}.  The bonus is awarded for questions which exactly and honestly represent the mind's present understanding and its desire for specific improvements of that understanding.
}
\end{enumerate}


\begin{center}\textbf{Homework}\end{center}

The weekly homework assignments for the next 6 weeks will be posted soon.  Each assignment is intended to require between 3 and 5 hours.  Please write the solutions in your notebook, to be collected Tuesdays and returned Thursdays.  Clear, correct, and complete solutions are graded.  Any absence of these 3 elements will be noted and you may resubmit your notebook until all elements are present.

\vspace{1pc}

Homework suggestions:
\begin{enumerate}
\item{Do not assume you already understand the topics.}
\item{Beware the substitution of conceptual understanding for technical proficiency or technical proficiency for conceptual understanding.}
\item{Solve a problem before planning its written solution.}
\item{If you find a clever or unusual way to solve a problem, decide whether it can be improved to a \emph{method} or whether it is merely a lucky guess.  Systematic methods can be used repeatedly and they may be useful in new contexts later on.}
\end{enumerate}

\end{document}