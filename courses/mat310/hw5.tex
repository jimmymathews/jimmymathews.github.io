\documentclass[12pt]{amsart}
\usepackage{amssymb,amsmath,amscd,graphicx,latexsym,amsthm,hyperref}
\usepackage{amssymb,latexsym,eufrak,amsmath,amscd,graphicx,hyperref}
\usepackage[all]{xy}

\usepackage{caption}
\usepackage{array}
\newcolumntype{L}[1]{>{\raggedright\let\newline\\\arraybackslash\hspace{0pt}}m{#1}}
\newcolumntype{C}[1]{>{\centering\let\newline\\\arraybackslash\hspace{0pt}}m{#1}}
\newcolumntype{R}[1]{>{\raggedleft\let\newline\\\arraybackslash\hspace{0pt}}m{#1}}

\setlength{\textwidth}{7.0 in}
\setlength{\topmargin} {-.3 in}
\setlength{\evensidemargin}{-.1 in}
\setlength{\oddsidemargin}{-.1 in}
\setlength{\footskip}{.2 in}
\setlength{\headheight}{.1 in}
\setlength{\textheight}{9 in}
\setlength{\parskip}{0pt}
\setlength{\parindent}{.2 in}
\RequirePackage[dvipsnames,usenames]{xcolor}
\hypersetup{
bookmarks,
bookmarksdepth=3,
bookmarksopen,
bookmarksnumbered,
pdfstartview=FitH,
colorlinks,backref,hyperindex,
linkcolor=Sepia,
citecolor=BlueViolet,
}
\usepackage{enumitem}

\usepackage{tikz,times,amsthm}

\begin{document}

\title{MAT 310 Linear Algebra Homework 5}
\maketitle


Section 2.3: 2a 3 9 11 12 13

\vspace{1pc}

(2a. omitted)

\vspace{1pc}

3.  Let $g(x)=3+x$, and define linear transformations $T:P_2(R)\rightarrow P_2(R)$ and $U:P_2(R)\rightarrow R^3$ by

\begin{align*}
T(f(x)) &= f'(x)g(x)+2f(x)\\
U(a+bx+cx^2) &= (a+b,c,a-b)
\end{align*}

Denote by $\beta$ and $\gamma$ the standard (ordered) bases of $P_2(R)$ and $R^3$.

\begin{enumerate}[label=(\alph*)]
\item{Compute directly the matrices of $U$, $T$, and $UT$ with respect to these bases.  Then verify the last one with the theorem on the composition of linear transformations (2.11 in your textbook). }
\item{Let $h(x)=3-2x+x^2$.  Compute the expression of $h(x)$ with respect to $\beta$ and the expression of $U(h(x))$ with respect to $\gamma$.  Then verify these expressions using the matrix of $U$ from the previous part and the theorem 2.14 in your textbook.}
\end{enumerate}

\vspace{1pc}

(a) By a direct calculation,
\begin{align*}
& T(\beta_1)= 0(3+x)+2(1)   &= 2(1) + 0(x) + 0(x^2)\\
& T(\beta_2)= 1(3+x)+2(x)   &= 3(1) + 3(x) + 0(x^2)\\
& T(\beta_3)= 2x(3+x)+2(x^2) &= 0(1)+ 6(x) + 4(x^2)
\end{align*}

The matrix $[T]_\beta$ of $T$ with respect to the basis $\beta$ is by definition the matrix whose columns are the expressions of the $T(\beta_i)$ with respect to the basis $\beta$.  Therefore this matrix is

\begin{align*}
[T]_\beta := \begin{bmatrix}
2 & 3 & 0\\
0 & 3 & 6\\
0 & 0 & 4\\
\end{bmatrix}
\end{align*}



By another direct calculation:
\begin{align*}
& U(\beta_1)= (1+0,0,1-0) &= (1,0,1) &= \gamma_1 + \gamma_3\\
& U(\beta_2)= (0+1,0,0-1) &= (1,0,-1)&= \gamma_1 - \gamma_3\\
& U(\beta_3)= (0, 1, 0)   &= (0,1,0) &= \gamma_2
\end{align*}

The matrix $[U]^{\gamma}_\beta$ of $U$ with respect to the bases $\beta$ and $\gamma$ is by definition the matrix whose columns are the expression of the $U(\beta_i)$ with respect to the basis $\gamma$.  Therefore this matrix is

\begin{align*}
[U]^{\gamma}_\beta := \begin{bmatrix}
1 & 1 & 0\\
0 & 0 & 1\\
1 & -1 & 0\\
\end{bmatrix}
\end{align*}

The last slew of direct calculations is:

\begin{align*}
& U(T(\beta_1))=U(2,0,0) = (2+0,0,2-0) = (2,0,2)\\
& U(T(\beta_2))=U(3,3,0) = (3+3,0,3-3) = (6,0,0)\\
& U(T(\beta_3))=U(0,6,4) = (0+6,4,0-6) = (6,4,-6)
\end{align*}

Therefore the matrix of $UT$ is

\begin{align*}
[UT]^{\gamma}_\beta := \begin{bmatrix}
2 & 6 & 6\\
0 & 0 & 4\\
2 & 0 & -6\\
\end{bmatrix}
\end{align*}

According to the theorem 2.11 from the textbook, this last matrix should equal to the following matrix product:

\begin{align*}
& [U]_{\beta}^{\gamma}[T]_{\beta} = \begin{bmatrix}
1 & 1 & 0\\
0 & 0 & 1\\
1 & -1 & 0\\
\end{bmatrix}\begin{bmatrix}
2 & 3 & 0\\
0 & 3 & 6\\
0 & 0 & 4\\
\end{bmatrix}\\
\end{align*}

This is true.  It can be verified by direct calculation of each entry according to the definition of the matrix product.

\vspace{1pc}

%Let $h(x)=3-2x+x^2$.  Compute the expression of $h(x)$ with respect to $\beta$ and the expression of $U(h(x))$ with respect to $\gamma$.  Then verify these expressions using the matrix of $U$ from the previous part and the theorem 2.14 in your textbook.

(b)  The expression of $h(x)=3-2x+x^2$ with respect to the standard basis $\beta$ of $P_2(R)$ is

\begin{align*}
& [h(x)]_\beta = \begin{bmatrix} 3 \\ -2 \\ 1 \end{bmatrix}
\end{align*}

According to a direct calculation, $U(h(x)) = (3+(-2),1,3-(-2)) = (1,1,5)$.  So the matrix of $U(h(x))$ with respect to the standard basis $\gamma$ of $R^3$ is

\begin{align*}
& [U(h(x)]_\gamma = \begin{bmatrix} 1 \\ 1 \\ 5 \end{bmatrix}
\end{align*}

According to the theorem in your textbook, these expressions should satisfy the matrix equation

\[[U]_{\beta}^{\gamma}[h(x)]_{\beta} = [U(h(x))]_{\gamma}\]

This can be verified by direct calculation of the entries of the matrix product shown below:

\begin{align*}
& \begin{bmatrix}
1 & 1 & 0\\
0 & 0 & 1\\
1 & -1 & 0\\ \end{bmatrix} \begin{bmatrix} 3 \\ -2 \\ 1 \end{bmatrix} = \begin{bmatrix} 1 \\ 1 \\ 5 \end{bmatrix}\\
\end{align*}

\noindent-----------------------------------------------------------------------------------------------------

\newpage

9.  Find linear transformations $U,T:F^2\rightarrow F^2$ such that $UT=0$ (the zero linear transformation), but $TU\neq 0$.  Use your answer to find matrices $A$ and $B$ such that $AB=O$ but $BA\neq O$.

\vspace{1pc}

Notice that the condition $UT=0$ is equivalent to the statement that the image of $T$ is contained in the null space of $U$.  So let's start with a transformation having null-space equal to the span of $(1,0)$ and another having image equal to this span.

For $(a,b)\in F^2$, set $U(a,b)=(b,b)$ and $T(a,b)=(a+b,0)$.  For each such $(a,b)$, 

\begin{align*}
& U(T(a,b)) = U(a+b,0) = (0,0) \\
\end{align*}

Therefore $UT = 0$.  Let's see if we also succeeded in arranging for $TU$ to be non-zero:

\begin{align*}
& T(U(a,b)) = T(b,b) = (b+b,0) = (2b,0) 
\end{align*}

We did.

Now, according to the theorem on composition of linear transformations, if $A$ denotes the matrix of $U$ and $B$ denotes the matrix of $T$ (all with respect to the standard basis of $F^2$, say), then the matrix product $AB$ should equal to the zero matrix $O$:

\begin{align*}
& \begin{bmatrix} 0 & 1\\ 0 & 1 \end{bmatrix} \begin{bmatrix} 1 & 1 \\ 0 & 0\end{bmatrix} = \begin{bmatrix} 0 & 0 \\ 0 & 0 \end{bmatrix}
\end{align*}

This is verified by direct calculation of the matrix product.

\noindent-----------------------------------------------------------------------------------------------------


11.  Let $T$ be a linear transformation.  Prove that $T^2 = 0$ if and only if $R(T)\subset N(T)$.

\vspace{1pc}

Suppose that the image of $T$ is contained in the null space of $T$.  By definition of the image, each $T(v)$ is in the image of $T$.  By the assumption, $T(v)$ is also in the null space of $T$.  Then by definition of the null space, $T(T(v))=0$.  Since this is true for all $v$ in the domain of $T$, it follows that $T^2=0$.

Now suppose that $T^2=0$.  Each element of the image of $T$ is of the form $T(v)$ for some $v$ in the domain of $T$.  By the assumption, $T(T(v)) = 0$.  So $T(v)$ also belongs to the null space of $T$.  Therefore the image of $T$ is contained in the null space of $T$.

\noindent-----------------------------------------------------------------------------------------------------

12.  Let $T$ and $U$ be linear transformations which are composable, i.e. so that the composition $UT$ is defined.

\begin{enumerate}[label=(\alph*)]
\item{Prove that if $UT$ is one-to-one, then $T$ is one-to-one.  Must $U$ also be one-to-one in this case?}
\item{Prove that if $UT$ is onto, then $U$ is onto.  Must $T$ also be onto in this case?}
\item{Prove that if $U$ and $T$ are one-to-one and onto, then $UT$ is also.}
\end{enumerate}

(a) Suppose that $UT$ is one-to-one.  According to a theorem in your textbook, this condition is equivalent to the statement that for each non-zero $x$ in the domain of $T$, $UT(x)$ does not equal to $0$.  For each such $x$, $T(x)$ is also not equal to zero (otherwise $U(T(x))$ would equal $U(0)=0$!).  Therefore $T$ is also one-to-one in this case.

However, in this case $U$ may not be one-to-one.  It is not hard to think of examples, but the idea is this: What can happen is that $U$ is one-to-one when restricted to the image of $T$, but has non-trivial null space elsewhere in its domain.

(b) Suppose that $UT$ is onto.  This means that, given a vector $x$ in the target space of $U$, there is a vector $y$ in the domain of $T$ such that $U(T(y))=x$.  In this case, there is also a vector $z$ such that $U(z)=x$, namely $z=T(y)$.  Therefore $U$ is also onto.

However, in this case $T$ may not be onto.  Again, it is not hard to think of examples, but the idea is: What can happen is that, even if the image of $T$ is not the entire domain of $U$, the restriction of $U$ to this image is already onto.

(c) Suppose that $U$ and $T$ are both one-to-one and onto.  Then for each non-zero $x$ in the domain of $T$, $T(x)$ is not equal to $0$, and further $U(T(x))$ is not equal to $0$.  Therefore $UT$ is one-to-one.  Also, for each $y$ in the target space of $U$, there is a vector $z$ such that $U(z) = y$.  Moreover, there is a vector $w$ such that $T(w)=z$.  For this $w$, $U(T(w))=U(z)=y$.  Therefore $UT$ is also onto.

\vspace{1pc}

(13 omitted)

\end{document}



















