% Everything on a line after the symbol % is a comment

%The document class defines the look and feel of the document
%Most of the times, you would use wither an article or a letter class
\documentclass[12pt]{article}

%This package enables the insertion of images into the document
\usepackage{graphicx}

%Enables AMS fonts and other useful commands
\usepackage{amsfonts}
\usepackage{amsmath,amssymb,amsthm}

%This sets 1.2-line spacing between the lines
\renewcommand{\baselinestretch}{1.2}

%These lines set margins, page size, etc. I don't remember
%most of them and use Google when I need to change anything
\setlength{\topmargin}{-0.7in}


\setlength{\textwidth}{6.5in}
\setlength{\oddsidemargin}{0.0in}
\setlength{\textheight}{9.1in}
\newlength{\pagewidth}
\setlength{\pagewidth}{6.5in}
\pagestyle{empty}




%These define 'custom commands'. Mostly, these are simply
%shortcuts for longer commands. The syntax is
%\def\blah{\longlonglongcommand}
%This means you would type \blah instead of \longlonglongcommand

%Paragraph without indentation
\def\pp{\par\noindent}

%Empty set
\def\O{{\varnothing}}

%=>
\def\>{{\Rightarrow}}

%Reals
\def\rr{\mathbf{R}}

%Rationals
\def\qq{\mathbf{Q}}


%Naturals
\def\nn{\mathbf{N}}

%Positive integers
\def\zz{\mathbf{Z^+}}

%emptyet
\def\ES{{\varnothing}}

%arrows
\def\>{{\Rightarrow}}
\def\=>{{\Rightarrow}}
\def\->{{\rightarrow}}
\def\M->{{\mapsto}}


\def\e{\exists}
\def\ne{\notexists}
\def\st{such that}
\def\s{\Sigma}
\def\d{\delta}
\def\e{\epsilon}

% |
\def\|{\mid}

%<=>
\def\lr{{\Leftrightarrow}}

%these are shortcuts for environment elements
\def\nums{\begin{enumerate}}
\def\nume{\end{enumerate}}
\def\x{\item}

%I have borrowed the following code from somebody. Note the usage later!

\newtheorem{theorem}{Theorem}[section]
\newtheorem*{theorem*}{Theorem}
\newtheorem{lemma}[theorem]{Lemma}
\newtheorem{proposition}[theorem]{Proposition}
\newtheorem{corollary}[theorem]{Corollary}
\newtheorem{exercise}[theorem]{Exercise}
\newenvironment{definition}[1][Definition]{\begin{trivlist}
\item[\hskip \labelsep {\bfseries #1}]}{\end{trivlist}}
\newenvironment{example}[1][Example]{\begin{trivlist}
\item[\hskip \labelsep {\bfseries #1}]}{\end{trivlist}}
\newenvironment{remark}[1][Remark]{\begin{trivlist}
\item[\hskip \labelsep {\bfseries #1}]}{\end{trivlist}}
\newenvironment{answer}[1][Answer]{\begin{trivlist}
\item[\hskip \labelsep {\bfseries #1}]}{\end{trivlist}}
\newenvironment{solution}[1][Solution]{\begin{trivlist}
\item[\hskip \labelsep {\bfseries #1}]}{\end{trivlist}}

\setcounter{secnumdepth}{-1} 

\usepackage{abstract}
\renewcommand{\abstractname}{}    % clear the title
\renewcommand{\absnamepos}{empty}


%Now let's enter data for the title page
\title{Actions, orbits, and Burnside's theorem}
\date{}
\begin{document}
\maketitle 
\begin{itemize}
\item
The rotation group of the icosahedron has 60 elements, and it is transitive on vertices, meaning that any vertex can be moved to any other by an appropriate symmetry.  There are 5 edges, arranged symmetrically, connected to any given vertex.  How many vertices are there?
\item
The rotation group of a tetrahedron has 12 elements (it is the alternating group $A_4$.)  There are 4 vertices.  What is the smallest rotation about a vertex appearing as a symmetry of the tetrahedron?
\end{itemize}

Consider a rotation $R$ of the cube of order 3, for example the element that we were representing last week as $(123)(546)$ (the faces of a standard die are numbered 1 through 6).
\begin{itemize}
\item
List all the elements of $\langle R\rangle$, the subgroup generated by $R$, and give an isomorphism $\langle R\rangle\rightarrow \mathbb{Z}_3$.
\item
This $G=\mathbb{Z}_3$ acts on the set of 8 vertices $X$.  Draw a picture of this set, grouped by orbit.  The orbit of a point $x$ under the action of $G$ is the set $\{ g\cdot x|g\in G\} $ of all points you can get $x$ to by applying symmetries.
\item
On a graph with the horizontal axis $X$ and the vertical axis $G=\mathbb{Z}_3$, circle all grid points $(x,g)$ such that $g\cdot x=x$.
\item
For each $x\in X$, determine $|G_x|$.
\item
For each $g\in G$, determine $|X^g|$.
\item
Compute $\sum_{x\in X}|G_x|/|G|$
\item
Compute $\sum_{g\in G}|X^g|/|G|$
\item
How many orbits are there?
\end{itemize}
\end{document}
